\chapter{\ifenglish Introduction\else บทนำ\fi}

\section{\ifenglish Project rationale\else ที่มาของโครงงาน\fi}

\section{\ifenglish Objectives\else วัตถุประสงค์ของโครงงาน\fi}
\hspace{0.9 cm}ศึกษาความเสถียรภาพของก๊าซคาร์บอนไดออกไซด์โดยใช้การประมวลผลภาพ (Image processing) เพื่อช่วยในการแยกแยะและนับจำนวนของคาร์บอนในการทดลองในห้องปฏิบัติการ

\section{\ifenglish Project scope\else ขอบเขตของโครงงาน\fi}
\begin{itemize}
    \item{การทดลองจะทดลองโดยก๊าซคาร์บอน}
    \item{ใช้การประมวลผลภาพในการนับจำนวนคาร์บอนในการทดลอง}
\end{itemize}

\subsection{\ifenglish Hardware scope\else ขอบเขตด้านฮาร์ดแวร์\fi}

\subsection{\ifenglish Software scope\else ขอบเขตด้านซอฟต์แวร์\fi}

\section{\ifenglish Expected outcomes\else ประโยชน์ที่ได้รับ\fi}
\hspace{0.9 cm}สามารถแยกก๊าซคาร์บอนและนับจำนวนคาร์บอนในการทดลองอีมัลชั่น

\section{\ifenglish Technology and tools\else เทคโนโลยีและเครื่องมือที่ใช้\fi}

\subsection{\ifenglish Hardware technology\else เทคโนโลยีด้านฮาร์ดแวร์\fi}
\begin{itemize}
    \item{Laptop computer ใช้ในการพัฒนาและทดสอบโค้ตในการนับจำนวนคาร์บอน}
    \item{Smartphone ใช้ในเก็บบันทึกข้อมูลการทดลอง}
\end{itemize}

\subsection{\ifenglish Software technology\else เทคโนโลยีด้านซอฟต์แวร์\fi}
\begin{itemize}
    \item{Virtual Studio Code ใช้ในการพัฒนาโค้ตในการนับจำนวนคาร์บอน}
    \item{Pycharm ใช้ในการพัฒนาโค้ตในการนับจำนวนคาร์บอน}
\end{itemize}

\section{\ifenglish Project plan\else แผนการดำเนินงาน\fi}

\begin{plan}{10}{2020}{3}{2021}
    \planitem{10}{2020}{10}{2020}{พูดคุยภายในกลุ่มเกี่ยวกับโครงงาน}
    \planitem{11}{2020}{1}{2021}{ศึกษาเกี่ยวกับอีมัลชั่น,การประมวลผลภาพ, Candy edge detection และ Active Contour : Snake Model}
    \planitem{12}{2020}{2}{2021}{จัดทำสไลด์นำเสนอ}
    \planitem{12}{2020}{2}{2021}{ตรวจทานและแก้ไขข้อผิดพลำด}
    \planitem{2}{2021}{2}{2021}{นำเสนอรอบที่ 1}
    \planitem{3}{2021}{3}{2021}{นำเสนอรอบที่ 2}
\end{plan}

\section{\ifenglish Roles and responsibilities\else บทบาทและความรับผิดชอบ\fi}
อธิบายว่าในการทำงาน นศ. มีการกำหนดบทบาทและแบ่งหน้าที่งานอย่างไรในการทำงาน จำเป็นต้องใช้ความรู้ใดในการทำงานบ้าง

\section{\ifenglish%
Impacts of this project on society, health, safety, legal, and cultural issues
\else%
ผลกระทบด้านสังคม สุขภาพ ความปลอดภัย กฎหมาย และวัฒนธรรม
\fi}

แนวทางและโยชน์ในการประยุกต์ใช้งานโครงงานกับงานในด้านอื่นๆ รวมถึงผลกระทบในด้านสังคมและสิ่งแวดล้อมจากการใช้ความรู้ทางวิศวกรรมที่ได้
