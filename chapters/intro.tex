\chapter{\ifenglish Introduction\else บทนำ\fi}

\section{\ifenglish Project rationale\else ที่มาของโครงงาน\fi}

\hspace{0.5 cm}การวิเคราะห์เสถียรภาพของอิมัลชัน เป็นสิ่งสำคัญในวิศวกรรมทรัพยากรธรณี เช่น เป็นที่นิยมในการขุดเจาะน้ำมันและก๊าซธรรมชาติ ความเสถียรของอิมัลชันเหล่านี้ มีผลต่อประสิทธิภาพการขุดเจาะ และความปลอดภัยของสิ่งแวดล้อม และการประมวลผลภาพ สามารถใช้เพื่อวิเคราะห์ และตรวจสอบเสถียรภาพของระบบทรัพยากรธรณี เช่น ป่าไม้, ทรัพยากรน้ำ, หรือ อื่น ๆ ที่เป็นที่สำคัญในวิศวกรรมทรัพยากรธรณี

การตรวจสอบสภาพแวดล้อม: การประมวลผลภาพสามารถใช้เพื่อตรวจสอบสภาพแวดล้อมของ \newline
ทรัพยากรธรณี เช่น การตรวจสอบป่าไม้เพื่อความหนาแน่นของต้นไม้, การตรวจสอบคุณภาพน้ำในลำแม่น้ำ, หรือการตรวจสอบการเปลี่ยนแปลงของทัพพีที่ใช้ในวิศวกรรมทรัพยากรธรณี

การควบคุมการใช้ทรัพยากร: การประมวลผลภาพช่วยในการควบคุมการใช้ทรัพยากรธรณีโดยการตรวจสอบปริมาณทรัพยากรที่ถูกใช้, เช่น การวิเคราะห์การใช้น้ำในเขตการเกษตรหรือการตรวจสอบการใช้พื้นที่ในการทำเหมืองแร่

การตรวจสอบการเปลี่ยนแปลง: การนำเข้าข้อมูลทางภูมิศาสตร์ และประมวลผลภาพช่วยในการตรวจสอบการเปลี่ยนแปลงของทรัพยากรธรณีตลอดเวลา เช่น การตรวจสอบการเติบโตของเมือง หรือการตรวจสอบการเปลี่ยนแปลงในพื้นที่ที่เป็นที่นิยมในการท่องเที่ยว

การควบคุมการปล่อยก๊าซ: การใช้การประมวลผลภาพเพื่อวิเคราะห์การปล่อยก๊าซคาร์บอน หรือสาร \newline
มลพิษในทรัพยากรธรณี, เช่น การตรวจสอบการปล่อยก๊าซคาร์บอนจากพื้นที่ที่มีการผลิตหรือการใช้เชื้อเพลิง

การประมวลผลภาพที่ใช้ในวิศวกรรมทรัพยากรธรณี มีความสามารถที่จะช่วยในการวิเคราะห์ข้อมูลทางภูมิศาสตร์และสร้างข้อมูลที่มีประสิทธิภาพเพื่อการบริหารจัดการทรัพยากรธรณีอย่างเป็นระบบ

ปัญหาที่พบคือ การวิเคราะห์บางอิมัลชัน ไม่สามารถทำได้ เช่น การวิเคราะห์ความเสถียรภาพของก๊าซคาร์บอน กลุ่มได้พิจารณาถึงวิธีแก้ไขปัญหานี้ และตกลงในการใช้ความรู้ที่มีสะสมและเรียนมาในการประยุกต์ใช้ในการวิเคราะห์ก๊าซคาร์บอน โดยกลุ่มเราจะใช้การประมวลผลภาพ ดังต่อไปนี้ เข้ามาช่วยในการวิเคราะห์ความเสถียรภาพของก๊าซคาร์บอน

\begin{enumerate}
    \item {การจำแนกก๊าซคาร์บอน: ใช้เทคนิคการประมวลผลภาพที่สามารถจำแนกก๊าซคาร์บอนในภาพได้ \newline
    โดยอาจใช้วิธี Canny edge detection เพื่อระบุและจำแนกก๊าซต่างๆ จากภาพ}
    \item {การนับจำนวนก๊าซคาร์บอน: ใช้เทคนิคการประมวลผลภาพที่สามารถนับจำนวนก๊าซคาร์บอนในรูป \newline
    ภาพได้ โดยอาจใช้ Active Contour : Snake Model เพื่อประมาณการจำนวนก๊าซ}
    \item {การวิเคราะห์รูปร่างของก๊าซคาร์บอน: หากปัญหาคือ การวิเคราะห์รูปร่างของก๊าซคาร์บอนที่มีลักษณะแตกต่างกัน ทีมวิจัยอาจต้องพัฒนาวิธีการที่สามารถจำแนกและวิเคราะห์รูปร่างที่แตกต่างของก๊าซ}
\end{enumerate}

เพื่อช่วยในการจำแนกและนับจำนวนก๊าซคาร์บอนที่มีรูปร่างต่างๆ ว่ามีจำนวนเท่าไหร่ โดยไม่ว่ารูปทรงของก๊าซคาร์บอนจะเป็นอย่างไร

\section{\ifenglish Objectives\else วัตถุประสงค์ของโครงงาน\fi}

\hspace{0.5 cm}ศึกษาความเสถียรภาพของก๊าซคาร์บอนไดออกไซด์โดยใช้การประมวลผลภาพ (Image processing) \newline
เพื่อช่วยในการแยกแยะและนับจำนวนของคาร์บอนในการทดลองในห้องปฏิบัติการ

\section{\ifenglish Project scope\else ขอบเขตของโครงงาน\fi}
\begin{itemize}
    \item {การทดลองจะทดลองโดยก๊าซคาร์บอน}
    \item {ใช้การประมวลผลภาพในการนับจำนวนคาร์บอนในการทดลอง}
\end{itemize}

\subsection{\ifenglish Hardware scope\else ขอบเขตด้านฮาร์ดแวร์\fi}

\hspace{0.5 cm}ผู้ทดลอง สามารถใช้การวิเคราะห์อีมัลชั่น ผ่านคอมพิวเตอร์

\subsection{\ifenglish Software scope\else ขอบเขตด้านซอฟต์แวร์\fi}

\hspace{0.5 cm}ระบบการวิเคราะห์อีมัลชั่นใช้งานผ่านเครื่องคอมพิวเตอร์

\section{\ifenglish Expected outcomes\else ประโยชน์ที่ได้รับ\fi}

\begin{itemize}
    \item {สามารถแยกก๊าซคาร์บอนและนับจำนวนคาร์บอนในการทดลองอีมัลชั่น}
    \item {ลดเวลาในการนั่งนับจำนวนก๊าซคาร์บอนในการทดลอง}
\end{itemize}

\section{\ifenglish Technology and tools\else เทคโนโลยีและเครื่องมือที่ใช้\fi}

\subsection{\ifenglish Hardware technology\else เทคโนโลยีด้านฮาร์ดแวร์\fi}
\begin{itemize}
    \item {Laptop computer ใช้ในการพัฒนาและทดสอบโค้ตในการนับจำนวนคาร์บอน}
    \item {Smartphone ใช้ในเก็บบันทึกข้อมูลจำนวนก๊าซคาร์บนอจากการทดลอง}
\end{itemize}

\subsection{\ifenglish Software technology\else เทคโนโลยีด้านซอฟต์แวร์\fi}
\begin{itemize}
    \item {Virtual Studio Code ใช้ในการพัฒนาโค้ตในการนับจำนวนคาร์บอน}
    \item {Pycharm ใช้ในการพัฒนาโค้ตในการนับจำนวนคาร์บอน}
    \item {Github ใช้ในการนำโค้ตที่เขียน pull ลงไป}
\end{itemize}


\begin{figure}[h!]
    \begin{center}
      \includegraphics[width=\textwidth]{Doc1.jpg}
    \end{center}
  \end{figure}


\section{\ifenglish Project plan\else แผนการดำเนินงาน\fi}

\begin{plan}{10}{2020}{3}{2021}
    \planitem{10}{2020}{10}{2020}{พูดคุยภายในกลุ่มเกี่ยวกับโครงงาน}
    \planitem{11}{2020}{1}{2021}{ศึกษาเกี่ยวกับอีมัลชั่น,การประมวลผลภาพ, Candy edge detection และ Active Contour : Snake Model}
    \planitem{12}{2020}{2}{2021}{จัดทำสไลด์นำเสนอ}
    \planitem{12}{2020}{2}{2021}{ตรวจทานและแก้ไขข้อผิดพลำด}
    \planitem{2}{2021}{2}{2021}{นำเสนอรอบที่ 1}
    \planitem{3}{2021}{3}{2021}{นำเสนอรอบที่ 2}
\end{plan}

\section{\ifenglish Roles and responsibilities\else บทบาทและความรับผิดชอบ\fi}

หน้าที่ในการทำโครงงาน

\begin{itemize}
    \item {แบ่งงาน และการนัดพูดคุยกันภายในกลุ่ม หรือนัดปรึกษาพูดคุยกับอาจารย์ที่ปรึกษา คนที่รับทำหน้าที่ คือ นายยศกร ลิขิตรังสรรค์}
    \item {ศึกษาหาความรู้เกี่ยวกับการเกิดอีมัลชั่น การประมวลภาพ วิธีการเขียนโค้ต คนที่รับทำหน้าที่ คือ นายยศกร ลิขิตรังสรรค์ , นายธนัญชัย ชัยมณี}
    \item {ศึกษาวิธีการนับจำนวนคาร์บอน ในรูปทรงต่างๆ คนที่รับทำหน้าที่คือ นายคเชนทร์ ไชโย}
    \item {เขียนโค้ตการประมวลผลภาพ การนับจำนวนคาร์บอน ที่มีรูปทรงต่างๆ  คนที่รับทำหน้าที่คือ ช่วยกันเขียนโค้ต}
    \item {การทดลองการเกิดอีมัลชั่นของก๊าซคาร์บอน  คนที่รับทำหน้าที่คือ ช่วยกันทำการทดลองโดยการสลับการเฝ้าการเกิดปฏิกิริยา}
\end{itemize}

\section{\ifenglish%
Impacts of this project on society, health, safety, legal, and cultural issues
\else%
ผลกระทบด้านสังคม สุขภาพ ความปลอดภัย กฎหมาย และวัฒนธรรม
\fi}

\hspace{0.5 cm}ผู้จัดทํามองว่า การที่โครงงานมีวัตถุประสงค์ที่จะวิเคราะห์ความเสถียรภาพของอีมัลชั่น จะช่วยยกระดับการทดลองทางภาคธรณีได้ แต่ในขณะเดียวกัน การวิเคราะห์นี้เกิดผลกระทบต่อการทดลองความปลอดภัยแก่ผู้ทดลองหาความเสถียรภาพอีมัลชั่นที่เป็นอันตรายต่อผู้ทดลองที่ต้องการนับจำนวนคาร์บอนกับก๊าซอื่นๆ เพราะก๊าซคาร์บอนทำปฏิกิริยากับการบางชนิดก็เป็นอันตรายอย่างร้ายแรง